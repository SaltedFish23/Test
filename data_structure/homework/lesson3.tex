\documentclass[12pt,a4paper,oneside]{ctexart}

\title{Lesson3 Practice}
\author{221900363 肖逸飞}

\usepackage{listings}
\usepackage{xcolor}
\definecolor{commentgreen}{RGB}{2,112,10}
\definecolor{eminence}{RGB}{108,48,130}
\definecolor{weborange}{RGB}{255,165,0}
\definecolor{frenchplum}{RGB}{129,20,83}


\begin{document}

\lstset {
    language=C++,
    frame=tb,
    tabsize=4,
    showstringspaces=false,
    numbers=left,
    %upquote=true,
    commentstyle=\color{commentgreen},
    keywordstyle=\color{eminence},
    stringstyle=\color{red},
    basicstyle=\small\ttfamily, % basic font setting
    emph={int,char,double,float,unsigned,void,bool},
    emphstyle={\color{blue}},
    escapechar=\&,
    % keyword highlighting
    classoffset=1, % starting new class
    otherkeywords={>,<,.,;,-,!,=,~},
    morekeywords={>,<,.,;,-,!,=,~},
    keywordstyle=\color{weborange},
    classoffset=0,
}

\maketitle

1.(1)pb != NULL
\\ (2)pa = pa->link
\\ (3)pb = pb->link
\\ (4)q->link = first->link
\\ (5) pa == NULL ? pb : pa
\\ (6) p != NULL
\\ (7) q->link = first->link
\\ (8) first->link = q

2.(1)AB*C*
\\ (2)BA-C-D+
\\ (3)0B-A*C+
\\ (4)AB+D*EFAD*+/+C+
\\ (5)AB\&\&EF>!||
\\ (6)ABC<CD>||!\&\&!CE<||

3.(1)132
\\ (2)435612:不可以,因为在完成4356的出栈后1和2都在栈内,此时2必然在1之前出栈
\\ 325641:可以,1进栈,2进栈,3进栈,3出栈,2出栈,4进栈,5进栈,5出栈,6进栈,6出栈,4出栈,1出栈
\\ 154623:不可以,在完成1546之后2和3必然在栈内,此时3必然在2之前出栈
\\ 135426:可以,1进栈,1出栈,2进栈,3进栈,3出栈,4进栈,5进栈,5出栈,4出栈,2出栈,6进栈,6出栈

4.IOIIOOIIOIIOIIOOOO

5.(1)

\begin{lstlisting}
    bool is_empty(){
        return  count == 0;
    }
    void in(int x){
        if(count < m){
            q[(front+count)%m] = x;
            count += 1;
        }
    }
    int out(){
        if(!is_empty()){
            int res = q[front];
            front = (front+1)%m;
            count -= 1;
            return res;
        }
    }
\end{lstlisting}

(2)m-1个

\end{document}